\documentclass[12pt]{article}

\usepackage{amsthm,amssymb,amsmath,amsfonts}
\usepackage[a4paper, top=25mm, bottom=30mm, left=25mm, right=25mm]{geometry}
\usepackage[pagebackref=false,colorlinks,linkcolor=black,citecolor=black]{hyperref}
\usepackage[nameinlink]{cleveref}
 \AtBeginDocument{%
    \crefname{equation}{برابری}{equations}%
    \crefname{chapter}{فصل}{chapters}%
    \crefname{section}{بخش}{sections}%
    \crefname{appendix}{پیوست}{appendices}%
    \crefname{enumi}{مورد}{items}%
    \crefname{footnote}{زیرنویس}{footnotes}%
    \crefname{figure}{شکل}{figures}%
    \crefname{table}{جدول}{tables}%
    \crefname{theorem}{قضیه}{theorems}%
    \crefname{lemma}{لم}{lemmas}%
    \crefname{corollary}{نتیجه}{corollaries}%
    \crefname{proposition}{گزاره}{propositions}%
    \crefname{definition}{تعریف}{definitions}%
    \crefname{result}{نتیجه}{results}%
    \crefname{example}{مثال}{examples}%
    \crefname{remark}{نکته}{remarks}%
    \crefname{note}{یادداشت}{notes}%
    \crefname{observation}{مشاهده}{observations}%
    \crefname{algorithm}{الگوریتم}{algorithms}%
    \crefname{cproof}{برهان}{cproofs}%
}

\usepackage{tikz}
\usepackage{graphicx}
\usepackage{color}

\usepackage{setspace}
\doublespacing

\usepackage{titletoc}
\usepackage{tocloft}
\usepackage{enumitem}

\usepackage{algorithm}
% \usepackage[noend]{algpseudocode}
\usepackage[noend]{algorithmic}
\renewcommand{\algorithmicrequire}{\textbf{Input:}}
\renewcommand{\algorithmicensure}{\textbf{Output:}}

\usepackage{tabularx}
\makeatletter
\newcommand{\multiline}[1]{%
  \begin{tabularx}{\dimexpr\linewidth-\ALG@thistlm}[t]{@{}X@{}}
    #1
  \end{tabularx}
}
\makeatother

\usepackage{float}
\usepackage{verbatim}
\makeindex
\usepackage{sectsty}
\usepackage{xepersian}
\SepMark{-}
\settextfont[Scale=1.2,Path=fonts/,BoldFont=B Nazanin Bold.ttf]{B Nazanin.ttf}
\setlatintextfont{Times New Roman}
\renewcommand{\labelitemi}{$\bullet$}

\theoremstyle{definition}
\newtheorem{definition}{تعریف}[section]
\newtheorem{remark}[definition]{نکته}
\newtheorem{note}[definition]{یادداشت}
\newtheorem{example}[definition]{نمونه}
\newtheorem{question}[definition]{سوال}
\newtheorem{remember}[definition]{یاداوری}
\newtheorem{observation}[definition]{مشاهده}
\theoremstyle{theorem}
\newtheorem{theorem}[definition]{قضیه}
\newtheorem{lemma}[definition]{لم}
\newtheorem{proposition}[definition]{گزاره}
\newtheorem{corollary}[definition]{نتیجه}
\newtheorem*{cproof}{برهان}
\usepackage{minted}
\usepackage{listings}
\usepackage{color}

\definecolor{dkgreen}{rgb}{0,0.6,0}
\definecolor{gray}{rgb}{0.5,0.5,0.5}
\definecolor{mauve}{rgb}{0.58,0,0.82}

\lstset{frame=tb,
  language=Java,
  aboveskip=3mm,
  belowskip=3mm,
  showstringspaces=false,
  columns=flexible,
  basicstyle={\small\ttfamily},
  numbers=none,
  numberstyle=\tiny\color{gray},
  keywordstyle=\color{blue},
  commentstyle=\color{dkgreen},
  stringstyle=\color{mauve},
  breaklines=true,
  breakatwhitespace=true,
  tabsize=3
}

\usepackage{listings}
\usepackage{xcolor}

\definecolor{codegreen}{rgb}{0,0.6,0}
\definecolor{codegray}{rgb}{0.5,0.5,0.5}
\definecolor{codepurple}{rgb}{0.58,0,0.82}
\definecolor{backcolour}{rgb}{0.95,0.95,0.92}

\lstdefinestyle{mystyle}{  
    commentstyle=\color{codegreen},
    keywordstyle=\color{magenta},
    numberstyle=\tiny\color{codegray},
    stringstyle=\color{codepurple},
    basicstyle=\ttfamily\footnotesize,
    breakatwhitespace=false,         
    breaklines=true,                 
    captionpos=b,                    
    keepspaces=true,                 
    numbers=left,                    
    numbersep=5pt,                  
    showspaces=false,                
    showstringspaces=false,
    showtabs=false,                  
    tabsize=2
}
\lstset{style=mystyle}


\begin{document}
\fontsize{12pt}{14pt}\selectfont

\begin{minipage}{0.1\textwidth}

\end{minipage}%
\hfill%
\begin{minipage}{0.6\textwidth}\centering
\fontsize{10pt}{10pt}\selectfont
به نام خداوند \\
ساختمان داده و الگوریتم \\
حل تمرین دوم\\
دکتر حاجی اسماعیلی \\
\vspace{0.25cm}
\begingroup
\fontsize{8pt}{8pt}\selectfont
دانشکده ریاضی و علوم کامپیوتر \\
فروردین ماه 1402 \\
\endgroup
\end{minipage}%
\hfill%
\begin{minipage}{0.1\textwidth}
\end{minipage}

\vspace{0.5cm}

\noindent\rule{\textwidth}{1pt}


\section{رفتار توابع}
در اینجا توابع بر اساس پیچیدگی های زمانی آنها به ترتیب صعودی مرتب شده اند:
\begin{itemize}
    \item $f8 = log(logn)$ 
    \item $f7=\sqrt{logn}$
    \item $f6 = n$
    \item $f5 = \binom{n}{n/2}$ به کمک تقریب استرلینگ و جایگذاری آن در فرم باز شده ی انتخاب میبینیم که این نیز یک پیچیدگی زمانی چند جمله‌ای است.
    \item $f2 = n^2 log(n^2023)$ این تابع با نرخی متناسب با $n^2$ رشد می کند، اما عبارت لگاریتمی نرخ رشد را با ضریب $log(n^(2023))$ = $2023*log(n)$ افزایش می دهد. سریعتر از توابع چند جمله ای است.
    \item $(logn)^2023$ 
    \item $f10 = 4^(3nlogn)$ 
    \item $f9 = 7^n^2$ این تابع به دلیل پایه نمایی 7 و توان $n^2$ سریعتر از توابع قبلی رشد می کند.
\end{itemize}



\section{پيچيدگي قطعات كد}
\subsection{}
\begin{itemize}
    \item پاسخ : $O(n^2)$
    راه حل: $n+n-1+...+2+1$ = $(n(n-1))/2$
    \item پاسخ : $O(sqrt(n))$
    راه حل: پیچیدگی زمانی کد ارائه شده را می توان با تجزیه و تحلیل تعداد تکرارهایی که حلقه اجرا می کند تعیین کرد.

    در هر تکرار، مقدار $"i"$ با $"j"$ افزایش می یابد که از 1 شروع می شود و در هر تکرار 1 مقدار افزایش می یابد. بنابراین، حلقه بار $"j"$ را اجرا می کند، جایی که $"j"$ کوچکترین عدد است به طوری که:

    \begin{equation}
    1 + 2 + 3 + \cdots + j \geq n
    \end{equation}

    و طبق این خواهیم داشت که:

    \begin{equation}
    j \times (j + 1) / 2 \geq n
    \end{equation}

    پس خواهیم داشت که:
    
    \begin{equation}
    j \geq (-1 + \sqrt{1 + 8n}) / 2
    \end{equation}

    از آنجایی که ما به پیچیدگی زمانی در بدترین حالت علاقه مندیم، می توانیم عوامل ثابت و اصطلاحات مرتبه پایین را نادیده بگیریم. بنابراین، پیچیدگی زمانی حلقه را می توان به صورت $O(sqrt(n))$ تقریب زد، زیرا عبارت غالب در فرمول $"j"$ جذر $"n"$ است.

    \item حلقه در مرحله اول $n$ بار در مرحله دوم $n/2$ بار و به همین صورت تعداد بار تکرار میشود که خب خواهیم داشت:
    \begin{equation}
    \O(s) = O(n+n/2+n/3+...+n/n)= n\mathlarger{\mathlarger{\sum}}_{i=0}^{n}\frac{1}{i}
    \end{equation}

    حال به حل خود سیگما میپردازیم:

    \begin{equation}
    {\mathlarger{\sum}}_{i=0}^{n}\frac{1}{i} \leq \int_{1}^{n} \frac{1}{x} \,dx = ln(x)
    \end{equation}

    \begin{equation}
    {\mathlarger{\sum}}_{x=1}^{n}\frac{1}{x+1} \leq \int_{1}^{n} \frac{1}{x} \,dx = ln(x) \leq {\mathlarger{\sum}}_{x=1}^{n}\frac{1}{x}
    \end{equation}

    پس پاسخ نهایی خواهد بود: $O(nlogn)$
\end{itemize}

\subsection{}
حلقه درونی به تعداد $i$ بار تکرار میشود و خب $i$ تا $n$ بالا میرود و خب حلقه درونی $n$ بار تکرار میشود و حلقه بیرونی نیز $log(n)$ بار تکرار میشود پس به طور کلی حلقه ها $O(nlogn)$ دفغه تکرار میشوند و صرفا لازم است درون حلقه ها $O(1)$ اجرا شود.

\section{طولاني ترين دنباله افزايشي}
برای حل این مساله از $dynamic programming$ استفاده میکنیم و $dp[i]$ را طول بلند ترین دنباله افزایشی منتهی به نقطه $i$ ام در نظر میگیریم و به صورت زیر کد خود $implement$ میکنیم:


\begin{lstlisting}[language=Python]
    
def LIS(nums):
    n = len(nums)
    dp = [1] * n
    
    for i in range(1,n):
        for j in range(i):
            if nums[j] < nums[i]:
                dp[i] = max(dp[i], dp[j]+1)
    return max(dp)
\end{lstlisting}

که خب مساله $n$ مرتبه تکرار و در هر مرحله $O(n)$ کار صورت میگیرد پس مساله از اردر $O(n^2)$ است.

\section{حداكثر مجموع غير مجاور}
کد این سوال را میتوان به صورت زیر نوشت:
\begin{lstlisting}[language=Python]
    
def MSNA(input):
    n = len(input)
    if n == 0:
        return 0
    if n == 1:
        return input[0]
    if n == 2:
        return max(input[0], input[1])
    dp = [0] * n
    dp[0] = input[0]
    dp[1] = max(input[0], input[1])
    for i in range(2, n):
        dp[i] max(dp[i-1], dp[i-2] + input[i])
    return max(dp)
\end{lstlisting}

\end{document}