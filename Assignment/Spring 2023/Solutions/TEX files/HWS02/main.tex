\documentclass[12pt]{article}
\usepackage{graphicx}
\graphicspath{ {./images/} }
\usepackage{amsthm,amssymb,amsmath,amsfonts}
\usepackage[a4paper, top=25mm, bottom=30mm, left=25mm, right=25mm]{geometry}
\usepackage[pagebackref=false,colorlinks,linkcolor=black,citecolor=black]{hyperref}
\usepackage[nameinlink]{cleveref}
 \AtBeginDocument{%
    \crefname{equation}{برابری}{equations}%
    \crefname{chapter}{فصل}{chapters}%
    \crefname{section}{بخش}{sections}%
    \crefname{appendix}{پیوست}{appendices}%
    \crefname{enumi}{مورد}{items}%
    \crefname{footnote}{زیرنویس}{footnotes}%
    \crefname{figure}{شکل}{figures}%
    \crefname{table}{جدول}{tables}%
    \crefname{theorem}{قضیه}{theorems}%
    \crefname{lemma}{لم}{lemmas}%
    \crefname{corollary}{نتیجه}{corollaries}%
    \crefname{proposition}{گزاره}{propositions}%
    \crefname{definition}{تعریف}{definitions}%
    \crefname{result}{نتیجه}{results}%
    \crefname{example}{مثال}{examples}%
    \crefname{remark}{نکته}{remarks}%
    \crefname{note}{یادداشت}{notes}%
    \crefname{observation}{مشاهده}{observations}%
    \crefname{algorithm}{الگوریتم}{algorithms}%
    \crefname{cproof}{برهان}{cproofs}%
}

\usepackage{tikz}
\usepackage{graphicx}
\usepackage{color}

\usepackage{setspace}
\doublespacing

\usepackage{titletoc}
\usepackage{tocloft}
\usepackage{enumitem}

\usepackage{algorithm}
% \usepackage[noend]{algpseudocode}
\usepackage[noend]{algorithmic}
\renewcommand{\algorithmicrequire}{\textbf{Input:}}
\renewcommand{\algorithmicensure}{\textbf{Output:}}

\usepackage{tabularx}
\makeatletter
\newcommand{\multiline}[1]{%
  \begin{tabularx}{\dimexpr\linewidth-\ALG@thistlm}[t]{@{}X@{}}
    #1
  \end{tabularx}
}
\makeatother

\usepackage{float}
\usepackage{verbatim}
\makeindex
\usepackage{sectsty}
\usepackage{xepersian}
\SepMark{-}
\settextfont[Scale=1.2,Path=fonts/,BoldFont=B Nazanin Bold.ttf]{B Nazanin.ttf}
\setlatintextfont{Times New Roman}
\renewcommand{\labelitemi}{$\bullet$}

\theoremstyle{definition}
\newtheorem{definition}{تعریف}[section]
\newtheorem{remark}[definition]{نکته}
\newtheorem{note}[definition]{یادداشت}
\newtheorem{example}[definition]{نمونه}
\newtheorem{question}[definition]{سوال}
\newtheorem{remember}[definition]{یاداوری}
\newtheorem{observation}[definition]{مشاهده}
\theoremstyle{theorem}
\newtheorem{theorem}[definition]{قضیه}
\newtheorem{lemma}[definition]{لم}
\newtheorem{proposition}[definition]{گزاره}
\newtheorem{corollary}[definition]{نتیجه}
\newtheorem*{cproof}{برهان}
\usepackage{minted}
\usepackage{listings}
\usepackage{color}

\definecolor{dkgreen}{rgb}{0,0.6,0}
\definecolor{gray}{rgb}{0.5,0.5,0.5}
\definecolor{mauve}{rgb}{0.58,0,0.82}

\lstset{frame=tb,
  language=Java,
  aboveskip=3mm,
  belowskip=3mm,
  showstringspaces=false,
  columns=flexible,
  basicstyle={\small\ttfamily},
  numbers=none,
  numberstyle=\tiny\color{gray},
  keywordstyle=\color{blue},
  commentstyle=\color{dkgreen},
  stringstyle=\color{mauve},
  breaklines=true,
  breakatwhitespace=true,
  tabsize=3
}

\usepackage{listings}
\usepackage{xcolor}

\definecolor{codegreen}{rgb}{0,0.6,0}
\definecolor{codegray}{rgb}{0.5,0.5,0.5}
\definecolor{codepurple}{rgb}{0.58,0,0.82}
\definecolor{backcolour}{rgb}{0.95,0.95,0.92}

\lstdefinestyle{mystyle}{  
    commentstyle=\color{codegreen},
    keywordstyle=\color{magenta},
    numberstyle=\tiny\color{codegray},
    stringstyle=\color{codepurple},
    basicstyle=\ttfamily\footnotesize,
    breakatwhitespace=false,         
    breaklines=true,                 
    captionpos=b,                    
    keepspaces=true,                 
    numbers=left,                    
    numbersep=5pt,                  
    showspaces=false,                
    showstringspaces=false,
    showtabs=false,                  
    tabsize=2
}
\lstset{style=mystyle}


\begin{document}
\fontsize{12pt}{14pt}\selectfont

\begin{minipage}{0.1\textwidth}

\end{minipage}%
\hfill%
\begin{minipage}{0.6\textwidth}\centering
\fontsize{10pt}{10pt}\selectfont
به نام خداوند \\
ساختمان داده و الگوریتم \\
حل تمرین دوم\\
دکتر حاجی اسماعیلی \\
\vspace{0.25cm}
\begingroup
\fontsize{8pt}{8pt}\selectfont
دانشکده ریاضی و علوم کامپیوتر \\
فروردین ماه 1402 \\
\endgroup
\end{minipage}%
\hfill%
\begin{minipage}{0.1\textwidth}
\end{minipage}

\vspace{0.5cm}

\noindent\rule{\textwidth}{1pt}


\section{صرفا مقایسس}
\subsection{}
\begin{itemize}
    \item $sort$ $Bubble$
    \\
            $Bubble sort$ عناصر کنار هم دیگه در یک آرایه را مقایسه می‌کند و اگر ترتیب اشتباهی داشته باشند، آن‌ها را تعویض می‌کند. این روند را تا زمانی که کل آرایه مرتب شود تکرار می کند.
            
            \begin{itemize}
                \item بهترین حالت
                \\
                اگر آرایه از قبل مرتب شده باشد، $Bubble sort$ فقط باید یک بار از آرایه عبور می کند که زمان $O(n)$ طول می‌کشد. این به این دلیل است که هیچ تعویضی لازم نیست و الگوریتم می تواند زودتر خاتمه یابد.
            
                \item بدترین حالت
                \\
                بدترین حالت: اگر آرایه به ترتیب معکوس باشد، که میشود $O(n^2)$
            \end{itemize}
            
    \item $sort$ $Insertion$
            \\
            $Insertion Sort$ هر عنصر از یک آرایه را در موقعیت صحیح خود در یک زیرآرایه مرتب شده قرار می دهد. این روند را تا زمانی که کل آرایه مرتب شود تکرار می کند.
            \begin{itemize}
                \item بهترین حالت
                \\
                 اگر آرایه از قبل مرتب شده باشد، مرتب سازی فقط نیاز به انجام یک جابه جایی در هر خانه را دارد که زمان $O(n)$ طول می کشد. این به این دلیل است که هیچ عنصری نیاز به جابجایی ندارد..
            
                \item بدترین حالت
                \\
                اگر آرایه در جهت معکوس باشد، $Insertion Sort$ باید $n-1$ جابه جایی برای هر $node$ انجام دهد و هر عنصر را به موقعیت صحیح خود در زیرآرایه مرتب شده منتقل کند، که زمان $O(n^2)$ طول می کشد.
            \end{itemize}
    \item $sort$ $Merge$
    \\
    $Merge Sort$ یک آرایه را به دو نیمه تقسیم می کند به صورت بازگشتی، $Node$ های نهایی را مرتب می کند و سپس دو نیمه را در یک آرایه مرتب شده ادغام می کند.
                \begin{itemize}
                \item بهترین حالت
                \\
                 بهترین حالت پیچیدگی زمانی $Merge Sort$ $O(n log n)$ است. این زمانی اتفاق می افتد که آرایه از قبل مرتب شده باشد.
            
                \item بدترین حالت
                \\
                بدترین حالت پیچیدگی زمانی $Merge Sort$ نیز $O(n log n)$ است. این زمانی اتفاق می افتد که هر فراخوانی بازگشتی آرایه را به دو نیمه با اندازه های مختلف تقسیم می کند و باعث می شود درخت فراخوانی های بازگشتی نامتعادل شود و یا اینکه آرایه به صورت کاهشی چیده شده باشد.
            \end{itemize}
\end{itemize}
\subsection{}
بر اساس تحلیل پیچیدگی زمانی، من استفاده از $Merge sort$ یا $Heap sort$ را برای مرتب‌سازی آرایه‌های بزرگ توصیه می‌کنم.ولی $Merge sort$ که برای آرایه‌های بزرگ کارآمد است به فضای بیشتری نیاز دارد، در حالی که $Heap sort$ به فضای اضافی نیاز ندارد و می‌تواند برای مرتب‌سازی $In place$ استفاده شود.


\section{$Merge sort$ عه بد}
همانطور که میدانیم الگوریتم $Merg Sort$ با وجود عملکرد خوب زمانی ولی از دیدگاه مصرف حافظه عملکزد افتضاحی دارد و برای همین الگوریتمی وجود دارد به نام $External sort$ که 2 یا 3 برابر کند تر است ولی حافظه مصرفی از $O(n)$ به $O(1)$ میرسد.
\\
\\
$External sort$ یک الگوریتم مرتب‌سازی است که زمانی استفاده می‌شود که داده‌هایی که قرار است مرتب شوند نمی‌توانند یکباره با هم در حافظه قرار بگیرند. در این سناریو، داده ها به تکه های کوچکتری تقسیم می شوند که می توانند در حافظه قرار گیرند، به صورت جداگانه مرتب شده و سپس با هم ادغام می شوند.
\\
\\
$External sort$ معمولاً شامل دو مرحله است:
\begin{enumerate}
    \item مرحله مرتب‌سازی: در مرحله مرتب‌سازی، داده‌ها از فایل ورودی خوانده می‌شوند و به تکه‌های کوچک‌تری تقسیم می‌شوند که سپس با استفاده از یک الگوریتم در حافظه مرتب می‌شوند. هنگامی که تکه های کوچکتر مرتب شدند، به عنوان فایل های موقت بر روی دیسک نوشته می شوند.
    \item فاز $Merge$: در طول مرحله $Merge$، فایل های موقت مرتب شده در یک فایل خروجی مرتب شده ترکیب می شوند. این کار با ادغام مکرر جفت فایل های موقت تا زمانی که یک فایل مرتب شده تولید شود، انجام می شود.
\end{enumerate}
\\
\\
برای تغییر $Merge sort$ به یک الگوریتم $External sort$، باید الگوریتم را طوری تغییر دهیم که داده‌هایی را که خیلی بزرگ هستند و نمی‌توانند یک‌باره در حافظه جای دهند، تغییر دهیم. این شامل تقسیم داده ها به قطعات کوچکتر، مرتب کردن آنها به صورت جداگانه در حافظه و سپس ادغام تکه های مرتب شده با هم به همان روشی است که در بالا توضیح داده شد. این فرآیند می تواند تا زمانی که کل مجموعه داده مرتب شود تکرار شود.


\section{سریع تر نمیشه}
در $DTM$، فرض می‌کنیم که هر الگوریتم مرتب‌سازی مبتنی بر مقایسه را می‌توان به عنوان یک درخت دودویی نشان داد، که در آن هر گره داخلی مقایسه‌ای بین دو عنصر و هر گره برگ نشان‌دهنده یک دنباله مرتب‌شده است.

ارتفاع درخت تصمیم برای یک الگوریتم مرتب‌سازی نشان‌دهنده بدترین تعداد مقایسه‌های مورد نیاز برای مرتب‌سازی توالی ورودی است. بنابراین، کران پایین برای مسئله مرتب‌سازی حداقل ارتفاع درخت تصمیم است که می‌تواند تمام دنباله‌های ورودی ممکن به طول $n$ را مرتب کند.

و اینکه حداقل ارتفاع درخت تصمیم برای مرتب سازی $n$ عنصر $Ω (nlogn)$ است. این بدان معناست که هر الگوریتم مرتب‌سازی مبتنی بر مقایسه باید حداقل $Ω(nlogn)$ مقایسه را در بدترین حالت انجام دهد تا $n$ عنصر را مرتب کند.

\end{document}