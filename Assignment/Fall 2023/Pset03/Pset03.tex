\documentclass[12pt]{article}
\usepackage{graphicx}
\graphicspath{ {./images/} }
\usepackage{amsthm,amssymb,amsmath,amsfonts}
\usepackage[a4paper, top=25mm, bottom=30mm, left=25mm, right=25mm]{geometry}
\usepackage[pagebackref=false,colorlinks,linkcolor=black,citecolor=black]{hyperref}
\usepackage[nameinlink]{cleveref}
 \AtBeginDocument{%
    \crefname{equation}{برابری}{equations}%
    \crefname{chapter}{فصل}{chapters}%
    \crefname{section}{بخش}{sections}%
    \crefname{appendix}{پیوست}{appendices}%
    \crefname{enumi}{مورد}{items}%
    \crefname{footnote}{زیرنویس}{footnotes}%
    \crefname{figure}{شکل}{figures}%
    \crefname{table}{جدول}{tables}%
    \crefname{theorem}{قضیه}{theorems}%
    \crefname{lemma}{لم}{lemmas}%
    \crefname{corollary}{نتیجه}{corollaries}%
    \crefname{proposition}{گزاره}{propositions}%
    \crefname{definition}{تعریف}{definitions}%
    \crefname{result}{نتیجه}{results}%
    \crefname{example}{مثال}{examples}%
    \crefname{remark}{نکته}{remarks}%
    \crefname{note}{یادداشت}{notes}%
    \crefname{observation}{مشاهده}{observations}%
    \crefname{algorithm}{الگوریتم}{algorithms}%
    \crefname{cproof}{برهان}{cproofs}%
}

\usepackage{tikz}
\usepackage{graphicx}
\usepackage{color}

\usepackage{setspace}
\doublespacing

\usepackage{titletoc}
\usepackage{tocloft}
\usepackage{enumitem}

\usepackage{algorithm}
% \usepackage[noend]{algpseudocode}
\usepackage[noend]{algorithmic}
\renewcommand{\algorithmicrequire}{\textbf{Input:}}
\renewcommand{\algorithmicensure}{\textbf{Output:}}

\usepackage{tabularx}
\makeatletter
\newcommand{\multiline}[1]{%
  \begin{tabularx}{\dimexpr\linewidth-\ALG@thistlm}[t]{@{}X@{}}
    #1
  \end{tabularx}
}
\makeatother

\usepackage{float}
\usepackage{verbatim}
\makeindex
\usepackage{sectsty}
\usepackage{xepersian}
\SepMark{-}
\settextfont[Scale=1.2,Path=fonts/,BoldFont=B Nazanin Bold.ttf]{B Nazanin.ttf}
\setlatintextfont{Times New Roman}
\renewcommand{\labelitemi}{$\bullet$}

\theoremstyle{definition}
\newtheorem{definition}{تعریف}[section]
\newtheorem{remark}[definition]{نکته}
\newtheorem{note}[definition]{یادداشت}
\newtheorem{example}[definition]{نمونه}
\newtheorem{question}[definition]{سوال}
\newtheorem{remember}[definition]{یاداوری}
\newtheorem{observation}[definition]{مشاهده}
\theoremstyle{theorem}
\newtheorem{theorem}[definition]{قضیه}
\newtheorem{lemma}[definition]{لم}
\newtheorem{proposition}[definition]{گزاره}
\newtheorem{corollary}[definition]{نتیجه}
\newtheorem*{cproof}{برهان}
\usepackage{minted}
\usepackage{listings}
\usepackage{color}

\definecolor{dkgreen}{rgb}{0,0.6,0}
\definecolor{gray}{rgb}{0.5,0.5,0.5}
\definecolor{mauve}{rgb}{0.58,0,0.82}

\lstset{frame=tb,
  language=Java,
  aboveskip=3mm,
  belowskip=3mm,
  showstringspaces=false,
  columns=flexible,
  basicstyle={\small\ttfamily},
  numbers=none,
  numberstyle=\tiny\color{gray},
  keywordstyle=\color{blue},
  commentstyle=\color{dkgreen},
  stringstyle=\color{mauve},
  breaklines=true,
  breakatwhitespace=true,
  tabsize=3
}

\usepackage{listings}
\usepackage{xcolor}

\definecolor{codegreen}{rgb}{0,0.6,0}
\definecolor{codegray}{rgb}{0.5,0.5,0.5}
\definecolor{codepurple}{rgb}{0.58,0,0.82}
\definecolor{backcolour}{rgb}{0.95,0.95,0.92}

\lstdefinestyle{mystyle}{  
    commentstyle=\color{codegreen},
    keywordstyle=\color{magenta},
    numberstyle=\tiny\color{codegray},
    stringstyle=\color{codepurple},
    basicstyle=\ttfamily\footnotesize,
    breakatwhitespace=false,         
    breaklines=true,                 
    captionpos=b,                    
    keepspaces=true,                 
    numbers=left,                    
    numbersep=5pt,                  
    showspaces=false,                
    showstringspaces=false,
    showtabs=false,                  
    tabsize=2
}
\lstset{style=mystyle}


\begin{document}
\fontsize{12pt}{14pt}\selectfont
\begin{minipage}{0.1\textwidth}

\end{minipage}%
\hfill%
\begin{minipage}{0.6\textwidth}\centering
\fontsize{10pt}{10pt}\selectfont
به نام خداوند \\
ساختمان داده و الگوریتم \\
حل تمرین دوم\\
دکتر حاجی اسماعیلی \\
\vspace{0.25cm}
\begingroup
\fontsize{8pt}{8pt}\selectfont
دانشکده ریاضی و علوم کامپیوتر \\
فروردین ماه 1402 \\
\endgroup
\end{minipage}%
\hfill%
\begin{minipage}{0.1\textwidth}
\end{minipage}

\vspace{0.5cm}

\noindent\rule{\textwidth}{1pt}


\section{سوال اول 3.0}
در زیر یک درخت $AVL$ با نام $T$ وجود دارد. عملیات $T.delete$ را برای $node$ 8 انجام دهید و پس از هر عملیات $rotation$ مورد نیاز که در طول عملیات انجام شد، درخت را رسم کنید.

\includegraphics[width=300]{etc/image.png}

\section{سوال دوم 3.0}
\begin{enumerate}
    \item $Binary-Tree$ زیر دارای ویژگی $height-balance$ نیست، اما ویژگی درخت جستجوی باینری را برآورده می کند، با این فرض که $key$ هر آیتم عدد صحیح خودش است. $key$های تمام گره هایی که ریشه زیر درخت $balance$ای نیستند را مشخص کنید و چولگی آنها را محاسبه کنید.
    
    \includegraphics[width=300]{etc/photo_2023-12-01_15-17-55.jpg}

    \item $insertion$ها و حذف‌های زیر را یکی پس از دیگری به ترتیب روی $T$ با اضافه کردن یا حذف یک $leaf$ و در عین حال حفظ ویژگی درخت جستجوی دودویی انجام دهید (ممکن است نیاز باشد یک $key$ به یک $leaf$ تبدیل شود).
    برای این قسمت از $rotation$ برای تعادل درخت استفاده نکنید. پس از هر عملیات درخت اصلاح شده را رسم کنید.
    
    $T.insert(2)$
    
    $T.delete(49)$
    
    $T.delete(35)$
    
    $T.insert(85)$
    
    $T.delete(84)$
\end{enumerate}


\section{سوال سوم 1.0}
برای مجموعه کلید زیر درخت جستجوی باینری با ارتفاغ 2و3و4و5 بکشید.

{1,4,5,10,16,17,21}



\section{سوال چهارم 2.0}
اثبات کنید که بدون توجه به $node$ آغازین در ارتفاع $h$ در درخت جستجوی باینری، تعداد $k$ فراخوانی موفق $TREE-Sucessor$ زمان $O(k+h)$ را صرف میکند.

\section{سوال پنجم 3.0}
درخت دودویی جست وجو را طوری تغییر دهید تا بتوانید $k$ امین عدد را در $O(logn)$ به دست آورد. این تغییر بر روی کدام بخش درخت (حافظه، زمان ، ...) اعمال می شود؟ مرتبه تغییر را مشخص کنید.

\section{سوال ششم 3.0}
امیر قرار است از پله های برج ایفل که تعدادشان $n$ تا است بالا برود. هر کدام از پلە ها ارتفاعͬ دارد و ارتفاع پلە ها نیز متمایز است. امیر روی پلە ای که قرار می گیرد به پایین می بیند و جمع می کند همه پلەهایی را که تا اینجا بالا آمده و ارتفاع از پلە ای که رویش کمتر است را محاسبه می کند. فرض کنید لیست همه پلە ها را به ترتیب داریم. الگوریتمی از مرتبه $O(nlogn)$ ارائه دهید که مجموع همه اعداد نوشته شده توسط امیر را بدست آورد.


\end{document}  
